\documentclass[]{article}
\usepackage{ctex}
%opening
\usepackage{hyperref}
\usepackage{graphicx}
\usepackage{xcolor}
\usepackage{listings}
\title{}
\author{}
\lstset{
	language=Python,
	breaklines=true, % 允许自动换行
	breakatwhitespace=true, % 在空白处换行
}
\begin{document}
	% TODO: \usepackage{graphicx} required
	\maketitle
	\tableofcontents \newpage
	\section{Text\_To\_Speech函数}
\lstset{
	basicstyle=\ttfamily\small, % 使用较小的字体
	keywordstyle=\color{blue},
	language=Python,
	breaklines=true, % 自动换行
	frame=single,
	showstringspaces=false,
	postbreak=\mbox{\textcolor{red}{$\hookrightarrow$}\space} % 换行后的标记
}


	
	\subsection{函数概述}
	
	\texttt{text2speech} 函数用于将文本转换为语音,并播放或保存为 WAV 文件。该函数支持中断检测、异步播放和音量调整等功能。
	
	\subsection{参数说明}
	
	\begin{itemize}
		\item \texttt{text=''}: 待转换的文本,默认为空字符串。
		\item \texttt{index=0}:\\
		Index=1000表示同步播放,先文本转语音,然后在运行到播放语音的那一行代码时,会阻塞在那一行代码,直到语音全部播完,才会继续运行后面的代码\\
		Index=0表示异步播放,先文本转语音,然后在运行到播放语音的那一行代码时,会新开一个进程用来播音,而主程序可以继续推进,去对下一句话进行文本转语音,因此上一句话的播音和下一句话的文本转语音是同时进行的,节省时间\\
		值得注意的是,上一句话播音的进程会被记录到STATUS.LastPlayProcessor变量,一旦检测到打断,就会杀死上一句话播音的进程,防止用户打断之后,机器人还一直播音不停下
		\item \texttt{is\_beep=False}:\\is\_beep表示现在要播音的这句话是不是过渡句
		过渡句例如: “我在”、 “大家有什么问题吗”
		在播音的同时,会把正在播音的这句话记录到
		STATUS.LAST\_BROAD\_WORDS变量,日志中会记
		录这句话,以及用户听到这句话之后,用户说的下一
		句话,用途是测试语音识别的准确性\\
		例如日志中可能会记录:\\
		时间: yyyy.mm.dd, hh:mm:ss\\
		上一句播音的话:玉兔号是我国的月球探测车
		语音识别结果:玉兔号是什么时候登上月球的?
		由于过渡句中不包含信息,记录日志时不能仅仅记录
		过渡句,还应该把上上句播音的话也记录下来
		例如,当最后一句播音的话是过渡句,日志中可能会
		记录:\\
		时间: yyyy.mm.dd, hh:mm:ss\\
		上一句播音的话:这就是我对航天机械臂的介绍。\\
		大家有什么问题吗\\
		语音识别结果:航天机械臂有多长?\\
		此处“这就是我对航天机械臂的介绍。 ”就是上上句播
		音的话,由于最后一句播音的话是过渡句,所以上上
		句播音的话也会被记录在日志里
		\item \texttt{wavfile=None}: 指定 WAV 文件路径,若提供,则直接播放该文件而不进行文本到语音的转换。
		\item \texttt{ignore\_interrupt=False}: 是否忽略中断,默认为 False。若为 True,则即使检测到中断也继续播放。
	\end{itemize}
	
	\subsection{功能描述}
	
	\begin{enumerate}
		\item 检查音频输出设备是否存在,若不存在则打印提示信息并返回。
		\item 若未指定 \texttt{wavfile},则调用外部脚本进行文本到语音的转换,并保存为 WAV 文件。
		\item 检查是否有上一次的播放进程,若存在且未完成,则根据 \texttt{ignore\_interrupt} 参数决定是否中断上一次的播放。
		\item 调整生成的 WAV 文件的音量,并保存为新的文件。
		\item 根据音频卡的 ID,选择合适的播放方式进行播放。
		\item 根据 \texttt{index} 参数,决定是进行同步播放还是异步播放。
	\end{enumerate}
	
	\subsection{注意事项}
	
	\begin{itemize}
		\item 该函数依赖外部脚本和 \texttt{ffmpeg} 工具进行文本到语音的转换和音量调整。
		\item 要用进程自带的kill方法而非直接kill命令来终止正在进行的播放,防止播音卡死。
		\item 异步播放时,需要留出短暂的时间间隔,以避免文件读写冲突。
	\end{itemize}

\subsection{执行过程的没理解的地方}
\begin{itemize}
	\item 
	\begin{lstlisting}[language=Python]
ttsproc=subprocess.Popen(["python3","/home/kuavo/catkin_dt/
src/voice_pkg/scripts/kedaxunfei_tts/test_host_3090_tts.py", text, savepath])
\end{lstlisting}这个文件运行起来什么功能?\\答:tts是text\_to\_speech的缩写,不必深究内容。
	\item ttsproc.poll()有什么功能(已解决)?\\答:在Python中,poll()方法是subprocess.Popen对象的一个方法,用于检查子进程是否已经结束。如果子进程已经结束,poll()会返回子进程的退出码,这通常是一个整数值;如果子进程尚未结束,poll()会返回None。
	\item playproc,ttsproc,STATUS.Last\_Play\_Processor都是干什么的,什么时候要注意'在连续播放多次时,需要注意资源的释放和进程的管理,避免服务卡死。'\\答:\textbf{假设text2speech函数正要播放一句话text:
	ttsproc 是用来把text文本转为语音(即合成音频文件)的进程
	playproc 是用来把音频文件播放出来的进程\\
	STATUS.Last\_Play\_Processor 是上一句话播音的进程
	为什么要记录这些进程?因为一旦识别到用户的打断信号,此时
	正在进行的一切 文本转语音 和 播音 进程都应该被杀死。 只有
	把这些进程全部都记录下来,在识别到用户的打断信号时, 才能
	把它们全都杀死}
\end{itemize}	
	\section{listenuser}
	\subsection{函数描述}
	
	\texttt{listenuser} 函数用于根据用户的声音输入或键盘输入来获取用户的指令或信息。该函数支持通过声音输入和文本输入两种方式。当系统检测到声音输入设备时,会通过声音识别接口获取用户的指令;若未检测到声音输入设备,则会提示用户通过键盘输入信息。
	\subsection{参数说明}
	\begin{itemize}
		\item \textbf{text} (默认值: \texttt{'ding'}): 预设的文本输入,用于声音识别接口的默认输入。
		\item \textbf{iter} (默认值: \texttt{1}): 指定声音识别过程中的迭代次数,用于控制识别的精度或尝试次数。
	\end{itemize}
	\subsection{不确定的地方}
\begin{itemize}
	\item \underline{iat\_web\_api是做什么的,参数什么意思?}
	\item \underline{ding有什么特殊的含义吗?}
\end{itemize}
\section{Pardon}

\texttt{pardon} 函数的目的是反复录制用户语音,直到获取到非空的语音输入为止,最多重复指定的轮数 \texttt{pardon\_round}。该函数在录制过程中提供了用户交互的反馈,并处理了用户可能的打断行为。

\subsection{参数说明}
\begin{itemize}
	\item \textbf{pardon\_round} (默认值: 1): 指定在放弃之前尝试录制用户语音的最大轮数。
\end{itemize}

\subsection{功能描述}
\begin{enumerate}
	\item 函数开始时,会打印一条消息提示开始录制用户语音。
	\item 通过 \texttt{listenuser} 函数反复录制用户语音,直到录制到非空的语音输入或达到最大重试次数。
	\item 如果录制到的语音为特定标记('\#\#\#\#'),或者是第一次提示用户提问,则通过 \texttt{text2speech} 函数反馈给用户,提示他们现在可以提问,并记录提示次数。
	\item 如果用户在6秒内没有说话,则会再次使用 \texttt{text2speech} 函数提示用户可以提问。
	\item 在用户提问过程中,如果检测到打断行为,会暂停QA(Question Answering)状态,并通过 \texttt{text2speech} 函数处理打断后的用户交互。
	\item 最后,无论是否录到有效语音,都会重置QA状态和打断状态,准备下一次录音或交互。
\end{enumerate}

\subsection{注意事项}
\begin{itemize}
	\item 该函数依赖于全局状态对象 \texttt{STATUS} 来管理QA状态和打断状态。
	\item 使用了 \texttt{listenuser} 和 \texttt{text2speech} 两个外部函数来分别处理语音输入和输出。
	\item 特殊标记 '\#\#\#\#' 用于特定逻辑处理,例如用户超时未回应时的处理。
\end{itemize}
\subsection{\underline{不理解的地方}}
\begin{itemize}
	\item 用户超过 6 秒没有说话就询问一遍在代码里哪里体现?
	\item \#\#\#\#是代填充的唤醒词吗?
	\item QAING这个状态有什么作用?
	\item 最后一个else是处理什么情况的?
\end{itemize}
	\section{QAClass}
	
	\texttt{QAClass} 是一个负责处理问答流程的类,它通过多线程初始化问答模型,并处理用户的提问,返回相应的答案。
	
	\subsection{构造函数}
	构造函数初始化了一个线程来启动问答模型的初始化过程,并设置了一些基本属性,如中断流的标志和任务类型标签列表。
	
	\subsection{QAClassInit}
	\texttt{QAClassInit} 方法用于初始化问答模型。它通过调用 \texttt{get\_llm\_answer} 函数获取模型实例,并将其存储在类的属性中。
	
	\subsection{answer\_question}
	\texttt{answer\_question} 方法是类的核心功能,负责处理用户的提问并返回答案。它首先等待问答模型初始化完成,然后根据用户的提问和其他相关信息,通过问答模型生成答案。此方法还处理了流式返回的答案,并在适当的时候通过语音合成技术将答案转换为语音输出。
	
	\subsection{\underline{我不理解的地方}}
\begin{itemize}
	\item \texttt{STATUS} 和 \texttt{get\_llm\_answer} 函数的具体实现和作用。(还包括 \texttt{llmclass} 的 \texttt{if\_document\_searched}, \texttt{process\_query} 这些)
	\item 816行,\texttt{text2speech("?")},这个问号怎么也能转语音
	\item \texttt{interrupt\_stream} 和 \texttt{STATUS} 里的 \texttt{interrupt} 什么区别,为什么要加上这个?
	\item 853行 \texttt{if self.interrupt\_stream:}\\
	\texttt{sleep(0.05)}\\
	\texttt{STATUS.set\_is\_Interrupted(False)}\\
	没看明白
\end{itemize}
\section{Interruption Class}
\subsection{continue\_rotate}
该函数用于控制机器人进行旋转以寻找人脸或手持麦克风的目标。函数接受一个参数 `rotate`,用于指定旋转方向。

\begin{itemize}
	\item \textbf{参数:}
	\begin{itemize}
		\item \texttt{rotate} (str): 指定旋转方向,默认为 'left'。可以是 'left'、'right' 或 'middle'。
	\end{itemize}
	\item \textbf{功能:}
	\begin{itemize}
		\item 如果 \texttt{rotate} 为 'middle',则打印 "此处不需要转向寻找人脸.",并且不进行旋转。
		\item 否则,打印 "转向寻找人脸ing...",并开始旋转。
	\end{itemize}
	\item \textbf{实现步骤:}
	\begin{enumerate}
		\item 设置发布频率为 30Hz。
		\item 创建一个 \texttt{Twist} 消息实例 \texttt{move\_cmd}。
		\item 初始化 \texttt{move\_cmd} 的角速度 \texttt{angular.z} 为 0.0,并发布该消息。
		\item 休眠一段时间以确保消息发布。
		\item 根据 \texttt{rotate} 参数设置旋转速度。如果 \texttt{rotate} 为 'left',速度为 0.5;否则速度为 -0.5。
		\item 将大人脸区域设置为 \texttt{None}。
		\item 进入一个循环,持续发布旋转命令,直到检测到手持麦克风并且手持麦克风标志被设置。
		\item 打印 "转向找到人脸"。
	\end{enumerate}
	\item \textbf{异常处理:}
	\begin{itemize}
		\item 在 \texttt{finally} 块中,停止旋转,将 \texttt{move\_cmd} 的角速度 \texttt{angular.z} 设置为 0.0,并发布该消息。重置状态
	\end{itemize}
	\item \textbf{\underline{一些不理解的地方:}}
	\begin{itemize}
		\item Twist的z的角速度是往哪里转向的?STATUS.HANDHELD\_DETECT and not STATUS.HANDHELD\_DETECT\_FLAG,这两个变量是谁设置的,什么含义?
		\item 
		STATUS.set\_Big\_Face\_Area(None) 什么叫Big\_Face\_Area
		\\答:Big\_Face\_Area:str = "LEFT",                     \# robot检测到的人脸区域
		\item 手持麦克风检测是怎么回事?哪里有手持麦克风?是检测到手持麦克风就是转到正对观众的方向了吗?
	\end{itemize}
\end{itemize}
\subsection{do\_action\_rely\_instruction}
该函数用于根据输入的问题执行相应的动作指令。函数接受一个参数 \texttt{question},用于指定动作指令的描述。

\begin{itemize}
	\item \textbf{参数:}
	\begin{itemize}
		\item \texttt{question} (str): 动作指令的描述,默认为空字符串。
	\end{itemize}
	\item \textbf{功能:}
	\begin{itemize}
		\item 根据输入的问题,确定动作类型和对应的动作索引。
		\item 如果无法确定动作类型,打印错误信息。
		\item 如果确定了动作类型,并且低计算机存在,发布动作指令并等待动作完成。
		\item 如果低计算机不存在,仅打印动作信息。
	\end{itemize}
	\item \textbf{实现步骤:}
	\begin{enumerate}
		\item 调用 \texttt{get\_action\_type} 函数获取动作类型。
		\item 初始化 \texttt{action\_index} 为 \texttt{None}。
		\item 遍历动作类型和对应的索引(前进、后退、左转、右转分别对应 101、102、103、104),如果动作类型匹配,设置 \texttt{action\_type} 和 \texttt{action\_index}。
		\item 如果 \texttt{action\_index} 仍为 \texttt{None},打印错误信息。
		\item 否则,检查 \texttt{STATUS.LOW\_COMPUTER\_EXIST} 是否为 \texttt{True}。
		\begin{itemize}
			\item 如果为 \texttt{True},打印执行动作信息,发布动作指令,并等待 \texttt{STATUS.NAVI\_END\_FLAG} 变为 'success'。
			\item 如果为 \texttt{False},仅打印执行动作信息。
		\end{itemize}
	\end{enumerate}
	\item \textbf{\underline{一些不懂的地方:}}
	\begin{itemize}
		\item  428行:\begin{verbatim}
			while True:
				if STATUS.NAVI_END_FLAG == 'success':
				STATUS.set_NAVI_END_FLAG(None)
				break
		\end{verbatim}这里的NAVI\_END\_FLAG看定义应该为bool类型,为什么这里还能和字符串比较?     
	\end{itemize}
\end{itemize}
\subsection{handle\_interrupt}
该函数用于处理所有情况下用户打断的操作。函数没有参数。

\begin{itemize}
	\item \textbf{功能:}
	\begin{itemize}
		\item 获取调用该方法的类的名称。
		\item 设置问答状态为正在进行。
		\item 如果检测到手持设备并且下位机存在,根据不同的类名称执行相应的旋转操作。
		\item 初始化任务类型为 \texttt{qa},并清空触摸屏目标点。
		\item 进入循环处理用户的指令,直到任务类型不再是 \texttt{qa}。
	\end{itemize}
	\item \textbf{实现步骤:}
	\begin{enumerate}
		\item 获取调用该方法的类的名称,并打印处理打断的信息。
		\item 设置问答状态为正在进行。
		\item 如果检测到手持设备并且低计算机存在,根据类名称执行相应的旋转操作。
		\item 初始化任务类型为 \texttt{qa},清空触摸屏目标点,并设置清除展品名称标志为 \texttt{True}。
		\item 进入循环处理用户的指令,直到任务类型不再是 \texttt{qa}。
		\begin{itemize}
			\item 如果满足条件,播放提示音“大家有什么问题吗?”或“我在”。
			\item 设置打断状态为 \texttt{False},并重置相关标志。
			\item 录制用户的指令。
			\item 如果用户没有点击触控屏,使用任务分类模型进行任务分类,并记录任务分类结果。
			\item 如果用户点击了触控屏,设置触摸屏目标点,并将任务类型设置为 \texttt{visit}。
			\item 如果启用了姿态检测,并且问题中没有明确指出展品名称,进行关键词识别,并根据用户回答更新问题。
			\item 如果任务类型仍为 \texttt{qa},生成答案并播音。如果启用了动作执行,启动动作执行线程并等待其完成。
		\end{itemize}
	\end{enumerate}
	\item \textbf{\underline{不理解的地方}}
	\begin{itemize}
		\item name = self.get\_self\_name() 什么时候能返回InterruptClass以及多个class?
		\item self.pre\_next\_situation = False这个属性是什么?
		\item STATUS.Touchpad\_Area == 'NowIsInterrupted'
\\		STATUS.set\_Touchpad\_Area('NowIgnoreTouchpad')这两种状态什么含义
	\end{itemize}
\end{itemize}
\end{document}
