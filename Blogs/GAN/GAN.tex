\documentclass[]{report}
\usepackage{xeCJK} % 支持中文
\usepackage{fontspec} % 设置字体
\usepackage{graphicx}
\usepackage{algorithm}
\usepackage{algorithmic}
\usepackage{bm} 
% 设置中文字体
\usepackage{placeins}
\usepackage{subcaption}
\usepackage{subcaption}
\usepackage{hyperref}
\usepackage{amsmath}
\usepackage{amssymb} % 或 \usepackage{amsfonts}
\setCJKmainfont{SimSun} % 使用宋体作为中文主字体
%opening
\title{Recent\_Note}
\author{LiXiaoLong}
\begin{document}
	
	\maketitle
	\tableofcontents
	\newpage
	\chapter{GAN}
	\section{简介}
	\subsection{设计思路}
	\begin{figure}[h]
		\centering
		\includegraphics[width=0.5\linewidth]{screenshot001}
		\caption{GAN设计思路}
		\label{fig:screenshot001}
	\end{figure}
	GAN 包括两个模型,一个是生成模型 \(G\)(Generator),一个是判别模型 \(D\)(Discriminator)。它们的功能分别是:
	
	\(G\) 负责生成图片,接收一个随机的噪声 \(z\),通过该噪声生成图片,记为 \(G(z)\)。
	
	\(D\) 负责判别一张图片是否“真实”。其输入是 \(x\),代表一张图片,输出 \(D(x)\) 表示 \(x\) 为真实图片的概率。输出为 1 代表真实图片的概率为 100\%,而输出为 0 则代表图片不可能是真实的(真实实例来源于数据集,伪造实例来源于生成模型)。
	
	\textbf{有一个很好的比喻,就是枯叶蝶的演化过程类似于树叶,枯叶蝶不需要认识树叶,但能通过变异逃避捕食者(筛选器),这样的自然选择使枯叶蝶越来越像树叶。同理,生成器产生的图片概率分布也会越来越接近真实数据集的概率分布。}
	
\subsection{损失函数和训练策略}
\subsubsection{(1)损失函数:}
	\begin{align}
		\min_G \max_D V(D, G) = & \mathbb{E}_{x \sim p_{\text{data}}(x)} [\log D(x)] 
		& + \mathbb{E}_{z \sim p_z(z)} [\log (1 - D(G(z)))]
	\end{align}
	含义很直接,对生成器,尽可能让$\mathbb{E}_{z \sim p_z(z)} [\log (1 - D(G(z)))]$更小,也就是$D(G(z))$尽可能大,前半段不涉及z,当常数处理。\\
	对判别器,就是真图像判别的结果越接近1越好,假图像越接近0越好。

\subsubsection{训练策略:}
\begin{figure}[hb]
	\centering
	\includegraphics[width=0.7\linewidth]{screenshot002}
	\caption{(2)训练过程}
	\label{fig:screenshot002}
\end{figure}
\indent Note:判别器是小蓝,真实图片的概率分布是小黑,生成器整的映射是下半边的箭头,由全数据区域到生成的图像空间,绿色是生成图像空间的概率分布。
\begin{itemize}
	\item 生成器和真实图像的概率分布偏差大,性能是比较差,判别器虽然大体在真是图像概率大的地方高,但是不稳定。
	\item 判别器被迭代了几轮,区分良好。
	\item 生成器被迭代更新,概率分布趋近了真实图像分布一些。
	\item 重复上面两步...
	\item 大结局,以假乱真,判别器out。
\end{itemize}
\subsection{数学推导}
推荐文章:https://www.cnblogs.com/LXP-Never/p/9706790.html
\section{Code}
\href{run:GAN.ipynb}{点击这里查看代码文件}
\subsubsection{(1):伪代码:}
\begin{algorithm}
	
	\begin{algorithmic}
		\label{alg:AGF}
		\FOR{number of training iterations}
		\FOR{$k$ steps}
		\STATE{$\bullet$ Sample minibatch of $m$ noise samples $\{ \bm{z}^{(1)}, \dots, \bm{z}^{(m)} \}$ from noise prior $p_g(\bm{z})$.}
		\STATE{$\bullet$ Sample minibatch of $m$ examples $\{ \bm{x}^{(1)}, \dots, \bm{x}^{(m)} \}$ from data generating distribution $p_\text{data}(\bm{x})$.}
		\STATE{$\bullet$ Update the discriminator by ascending its stochastic gradient:
			\[
			\nabla_{\theta_d} \frac{1}{m} \sum_{i=1}^m \left[
			\log D\left(\bm{x}^{(i)}\right)
			+ \log \left(1-D\left(G\left(\bm{z}^{(i)}\right)\right)\right)
			\right].
			\]}
		%parameters $\theta_d$ of discriminator $D$
		%in the direction of the stochastic gradient of the binomial cross-entropy
		%for $D$ predicting whether its argument comes from $p_\text{data}(\bm{x})$ (target = 1, input = $\bm{x}$) or
		%$P_g$ (target = 0, input = $G(\bm{z})$), i.e., towards minimizing
		% \mbox{$-\log D(\bm{x}) - \log(1 - D(G(\bm{z})))$}.}
	\ENDFOR
	\STATE{$\bullet$ Sample minibatch of $m$ noise samples $\{ \bm{z}^{(1)}, \dots, \bm{z}^{(m)} \}$ from noise prior $p_g(\bm{z})$.}
	\STATE{$\bullet$ Update the generator by descending its stochastic gradient:
		\[
		\nabla_{\theta_g} \frac{1}{m} \sum_{i=1}^m
		\log \left(1-D\left(G\left(\bm{z}^{(i)}\right)\right)\right)
		.
		\]}
	\ENDFOR
	\\The gradient-based updates can use any standard gradient-based learning rule. We used momentum in our experiments.
\end{algorithmic}
\end{algorithm}
\begin{figure}[htbp]
	\centering
	\begin{subfigure}[b]{0.4\linewidth}
		\includegraphics[width=\linewidth]{./images/0.png}
		\caption{训练0个epoch的生成结果}
		\label{fig:epoch0}
	\end{subfigure}
	\hfill % 添加一些水平空间
	\begin{subfigure}[b]{0.4\linewidth}
		\includegraphics[width=\linewidth]{./images/200.png}
		\caption{训练200个epoch生成的结果}
		\label{fig:epoch200}
	\end{subfigure}
	\caption{训练不同epoch数的生成结果对比}
	\label{fig:epochs_comparison}
\end{figure}
\chapter{VAE model}
\section{设计思路}
\subsection{Auto Encoder结构}
\begin{figure}[h]
	\centering
	\includegraphics[width=0.7\linewidth]{images/screenshot006}
	\caption{Auto Encoder结构}
	\label{fig:screenshot006}
\end{figure}
\indent VAE结构基于AE,这种网络结构有一个很经典的功能,数据降维,可以把高维度的数据变成很低的维度,但是仍然保留大部分语义特征(因为它能通过一个固定的函数(解码器)以比较小的误差恢复出来)。至于这个网络能干什么,去噪,数据压缩之类的。
\begin{itemize}
	\item 这时候刚好来引入一个概念,\textbf{latent space}(我就叫它表示空间好了),也就是图中z向量所在的空间,
	\item 第二个概念是,latent space的正则性,这里用两幅图描述。
	\begin{figure}[t]
		\centering
		\includegraphics[width=1\linewidth]{images/screenshot007}
		\label{fig:screenshot007}
	\end{figure}
	\begin{figure}[h]
		\centering
		\includegraphics[width=1\linewidth]{images/screenshot008}
		\caption{正则化的直观感受}
		\label{fig:screenshot008}
	\end{figure}
	
	\FloatBarrier
	这里的左侧的表示空间和右侧比较就没什么结构,差别很大的东西在表示空间里的位置却没有散开,相似的没有聚在一堆。
\end{itemize}
\subsection{灵感乍现,这个decoder似乎有做图像生成的潜力}
\indent 它能用低维度的一些看似随机的向量整出来有意义的图片,如果我们直接随机生成一些向量输进去会不会也能搞出来新的作品?不过先考虑下面的要求。
\begin{enumerate}
	\item 随机的向量都能有图片对应,这说明一个事情,latent space的大部分空间都会在高维空间得有有意义的图像对应。我们原有的训练策略从没有对表示空间做任何要求,它可以就好多离得很远的片区有有意义的图像对应。
	\item 它一个输入绝对只有一个输出,这不好,我们应该生成多个类似的但又不一样的图像以供选择。
\end{enumerate}
\subsection{VAE怎么满足这些要求的?}
\subsubsection{VAE结构}
\begin{figure}[h]
	\centering
	\includegraphics[width=0.8\linewidth]{images/screenshot012}
	\caption{VAE和AE区别}
	\label{fig:screenshot012}
\end{figure}
\FloatBarrier
它的编码器不再是和表征空间的一个点对应了,而是一个概率分布,解码器也是得到的一个概率分布,不过方差是提前固定的
\begin{figure}[ht]
	\centering
	\begin{subfigure}[b]{0.4\textwidth}
		\includegraphics[width=\textwidth]{images/screenshot009}
		\caption{VAE编码器}
		\label{fig:screenshot009}
	\end{subfigure}
	\hfill % 添加一些水平空间或者使用 \quad, \qquad 等来调整子图之间的间距
	\begin{subfigure}[b]{0.4\textwidth}
		\includegraphics[width=\textwidth]{images/screenshot010}
		\caption{VAE解码器}
		\label{fig:sub2}
	\end{subfigure}
	\begin{subfigure}[b]{0.8\textwidth}
		\includegraphics[width=\textwidth]{images/screenshot011}
		\caption{VAE整体结构}
		\label{fig:sub3}
		\end{subfigure}
	\caption{VAE结构}
	\label{fig:test}
\end{figure}
\FloatBarrier
\begin{itemize}
	\item 图a:解码器介绍:两部分构成的神经网络(前半边公用),输入是图像,输出是确定一种分布所需的参数(高斯分布为例,均值向量和协方差矩阵,不过为了减少计算量,假定各个特征也不相关,就成了方差向量了)
	\item 图b:编码器介绍:输入时从前面的分布从latent space采样的特征表示,输出是高斯分布的均值,实际上也就是输出图像的预期,有一个固定的方差,采样得到输出结果。
	\item 图c:整合的小策略:为了能让网络可以训练,特意用N(0,1)这个模块来帮助表示各种高斯分布,完成采样。
	\begin{figure}[h]
		\centering
		\includegraphics[width=0.7\linewidth]{images/screenshot013}
		\label{fig:screenshot013}
	\end{figure}
	
\end{itemize}
\subsubsection{进入正题}
VAE巧妙地用了概率分布作为编解码器映射的对象,而且选用高斯分布,训练时每一个真实样本,相当和表征空间的某个点为中心的一片区域对应上了,而且损失函数还要让这个概率分布接近$N(0,1)$,意味着映射的区域,或者点云都集中在一块,但又有一定的距离来确保准确性($\hat{x}-x$这个损失限制)

\section{损失函数怎么设计出来的?}
这里就进入了数学部分。\\
先定义
\begin{itemize}
	\item $x$:模型输入
	\item $z$:中间层的表示,在这个模型中是latent space采样后的结果
	\item $g(x)$: 输入:同编码器输入,输出:z所满足的高斯分布的均值
	\item $h(x)$: 输入:同编码器输入,输出:z所满足的高斯分布的方差
	\item $q_x(z)$ :  z的概率密度分布,$q_x(z) \equiv  \mathcal{N}(g(x),h(x)),g(x)\in G,h(x)\in  H;$\\ G H是函数族
\end{itemize}
再假设
\begin{itemize}
	\item $z$应该满足$\mathcal{N}$(0,1)
	\item $p(x|z)$也满足正态分布,方差是固定的,均值是通过解码器给出的。\,即$p(x|z)=\mathcal{N}(f(z),c)$
\end{itemize}
\newpage
$p(z|x)$根据上述假设,没法直接求出来,不过可以用正态分布函数族拟合,g和h是为了整出来一个正态分布来拟合$p(z|x)$
\begin{figure}[h]
	\centering
	\includegraphics[width=0.9\linewidth]{images/screenshot014}
	\label{fig:screenshot014}
\end{figure}
\FloatBarrier
又因为解码器还要满足
\begin{figure}[h]
	\centering
	\includegraphics[width=0.4\linewidth]{images/screenshot015}
	\label{fig:screenshot015}
\end{figure}
就是得选一个好的解码器f能让$\hat{x}=x$的概率最高。\\
综合一下,就是要优化这个:\begin{figure}[h]
	\centering
	\includegraphics[width=0.7\linewidth]{images/screenshot016}
	\label{fig:screenshot016}
\end{figure}
\FloatBarrier
证明完毕

\end{document}